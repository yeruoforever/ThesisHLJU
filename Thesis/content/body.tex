\section{学位论文的内容要求}

硕士学位论文内容一般应由十一个主要部分组成,依次为:封面→中文摘要→英文摘要→目录→符号说明→论文正文→参考文献→附录→致谢→攻读学位期间发表的学术论文→独创性声明。

博士学位论文内容一般应由十四个主要部分组成,依次为:封面→中文题目→英文题目→中文摘要→英文摘要→中文目录→英文目录→符号说明→论文正文→参考文献→附录→致谢→攻读学位期间发表的学术论文→独创性声明。

\subsection{题目}
题目应恰当、准确地反映本课题的研究内容。
学位论文的中文题目应不超过25个字,并且不设副标题。

\subsection{中文摘要与关键词}

\subsubsection{中文摘要}
中文摘要是论文内容的简要陈述,是一篇具有独立性和完整性的短文。
中文摘要应包括本论文的基本研究内容、研究方法、创造性成果及其理论与实际意义。
中文摘要中不宜使用公式、图表,不标注引用文献编号。避免将中文摘要写成目录式的内容介绍。

\subsubsection{关键词}
关键词是供检索用的主题词条,应采用能覆盖论文主要内容的通用技术词条(参照相应的技术术语标准)。
关键词一般列3~5个,按词条的外延层次从大到小排列。

\subsection{论文正文}
论文正文包括绪论、论文主体及结论等部分。

\subsubsection{绪论}
学位论文为了反映已掌握坚实的基础理论和广博的专门知识,对研究方案作了充分论证,因此,有关历史回顾和前人工作的综合评述,以及理论分析等,可以单独成章,有足够的文字论述。

博士学位论文要求有1万字左右的文献综述。

硕士学位论文要求有3000字左右的文献综述。

\subsubsection{论文主体}
论文主体是学位论文的主要部分,应该结构合理,层次清楚,重点突出,文字简练、通顺。
论文主体的内容应包括以下各方面:
\begin{itemize}
    \item 本研究内容的总体方案设计与选择论证;
    \item 本研究内容各部分(包括硬件与软件)的设计计算;
    \item 本研究内容的理论分析。
\end{itemize}

对本研究内容及成果应进行较全面、客观的理论阐述,应着重指出本研究内容中的创新、改进与实际应用之处。理论分析中,应将他人研究成果单独书写,并注明出处,不得将其与本人提出的理论分析混淆在一起。对于将其他领域的理论、结果引用到本研究领域者,应说明该理论的出处,并论述引用的可行性与有效性。


本研究内容各部分(包括硬件与软件)的设计计算;
本研究内容试验方案设计的可行性、有效性以及试验数据处理及分析;
本研究内容的理论分析。对本研究内容及成果应进行较全面、客观的理论阐述,应着重指出本研究内容中的创新、改进与实际应用之处。理论分析中,应将他人研究成果单独书写,并注明出处,不得将其与本人提出的理论分析混淆在一起。对于将其他领域的理论、结果引用到本研究领域者,应说明该理论的出处,并论述引用的可行性与有效性。

管理和人文学科的论文应包括对研究问题的论述及系统分析,比较研究,模型或方案设计,案例论证或实证分析,模型运行的结果分析或建议、改进措施等。

自然科学的论文应推理正确,结论明确,无科学性错误。

论文主体各章后一般应有“本章小结”。

\subsubsection{结论}
可以总结性地说明学位论文的最终研究成果及其价值,同时,可以直接表明论文中尚待进一步研究的不足之处。

结论应当体现作者更深的认识,且是从全篇论文的全部材料出发,经过推理、判断、归纳等逻辑分析过程而得到的新的学术总观念、总见解。

结论不应当是正文中各段小结的简单重复,结论内容一般在2000字以内。

\subsection{参考文献}
博士学位论文的参考文献数一般应不少于100篇,其中外文文献一般不少于总数的1/2;硕士学位论文的参考文献一般应不少于60篇,其中外文文献一般不少于30篇。参考文献中近五年的文献数一般应不少于总数的1/3,并应有近两年的参考文献。
\subsubsection{博士}
博士学位论文要求注释使用页下注,如果没有特别要求,应当以页为单位连续编号,以便于读者查阅,编号用数字加圈标注,如\textcircled{1}\textcircled{2}等。书写格式与参考文献列表的格式一致。

博士学位论文在文后列参考文献表,包括参考书目、参考著作、参考文章、外文文献等(必要时可分类列表)。参考文献序号左顶格用方括号标注,如[1][2]等。
\subsubsection{硕士}

硕士学位论文的参考文献是作者写论文时所参考、引用的文献书目,引用文献的标示应置于所引内容最后一个字的右上角,所引文献编号用阿拉伯数字置于方括号“[ ]”中,用小4号字体的上角标,如“二次铣削。”[1]不得将引用文献标示置于各级标题处。

硕士学位论文参考文献按正文中出现的先后顺序列表于文后(必要时可分类列表)。参考文献序号左顶格用方括号标注,如[1][2]等。具体内容参见第2章2.13参考文献。

\subsection{攻读学位期间发表的学术论文(博士必须有该项内容)}
学位论文后应列出研究生在攻读学位期间发表的与学位论文内容相关的学术论文(含已录用,并有录用通知书的学术论文。录用通知书中应明确说明论文的发表卷、期号。)。攻读学位期间所获得的科研成果可单做一项列出。与学位论文无关的学术论文不宜在此列出。我校对博士生攻读学位期间发表学术论文的要求见《黑龙江大学关于研究生学位论文答辩资格审查暂行规定》。
\subsection{致谢}
对导师和给予指导或协助完成学位论文工作的组织和个人表示感谢。内容应简洁明了、实事求是。对课题给予资助者应予感谢。
\pagebreak

\section{书写规定}
\subsection{论文正文字数}

博士学位论文理工科一般为8-10万字,其中绪论要求为1万字左右。

硕士学位论文理工科一般为2万字左右,其中绪论要求为3000字左右。

\subsection{论文书写}
研究生学位论文一律要求在计算机上输入、编排与打印。A4纸页边距设置为上4cm、下3.5cm、内侧3cm、外侧3cm。版式页眉和页脚设置为奇偶页不同,页眉3cm、页脚2.75cm。

论文版芯大小一般应为150mm×215mm(包括页眉及页码则为150mm×235mm),页码在版芯下边线之下隔线居中放置;摘要、目录、物理量名称及符号表等前置部分页码用大写罗马数字单独编排,如:I,II,Ⅲ等。正文页码按阿拉伯数字连续编排,如:1,2,3等。

博士学位论文的扉页、摘要、目录、图题及表题等,都要求用中、英文两种文字给出,编排上中文在前。扉页、摘要及目录的英文部分另起一页。

\subsection{中文摘要}
中文摘要的字数,硕士学位论文不宜超过300字,博士学位论文不宜超过500字,均以能将规定内容阐述清楚为原则。中文摘要页不需要写出论文题目。

英文摘要与中文摘要的内容应完全一致,在语法、用词上应准确无误。

\subsection{目录}
目录应包括论文中全部章、节、条三级标题及其页号,含:
\begin{itemize}
    \item 中文摘要
    \item Abstract
    \item 目录
    \item 物理量名称及符号表(参照附录1。采用国家标准规定符号者可略去此表)
    \item 正文章节题目(要求编到第3级标题,即×.×.×。一级标题顶格书写,二级标题缩进一格,三级标题缩进两格。)
    \item 参考文献
    \item 附录
    \item 致谢
    \item 攻读学位期间发表论文
    \item 独创性声明
    \item 索引(可选择或不选择)
\end{itemize}

\subsection{论文正文}

\subsubsection{章节及各章标题}
论文正文分章节撰写,每章应另起一页。

各章标题要突出重点、简明扼要。字数一般应在15字以内,不得使用标点符号。标题中尽量不采用英文缩写词,对必须采用者, 应使用本行业的通用缩写词。
\subsubsection{层次}

层次以少为宜,根据实际需要选择。层次代号建议采用文3.7中表1的格式。

层次要求统一,但若节下内容无需列条的,可直接列款、项。具体用到哪一层次视需要而定。

\subsection{名词术语}
科技名词术语及设备、元件的名称,应采用国家标准或部颁标准中规定的术语或名称。标准中未规定的术语要采用行业通用术语或名称。全文名词术语必须统一。一些特殊名词或新名词应在适当位置加以说明或注解。

采用英语缩写词时,除本行业广泛应用的通用缩写词外,文中第一次出现的缩写词应该用括号注明英文原词。
\subsection{物理量名称、符号与计量单位}
\subsubsection{物理量名称和符号}
物理量的名称和符号应符合GB3100~3102-86的规定(参照附录2)。论文中某一量的名称和符号应统一。

物理量的符号必须采用斜体。表示物理量的符号作下标时也用斜体。
\subsubsection{物理量计量单位}
物理量计量单位及符号应按国务院1984年发布的《中华人民共和国法定计量单位》(参照附录3)及GB3100~3102执行,不得使用非法定计量单位及符号。计量单位可采用汉字或符号,但应前后统一。计量单位符号,除用人名命名的单位第一个字母用大写之外,一律用小写字母。

非物理量单位(如件、台、人、元、次等)可以采用汉字与单位符号混写的方式,如“万t·km”,“t/(人·a)”等。

文稿叙述中不定数字之后允许用中文计量单位符号,如“几千克至1000kg”。

表达时刻时应采用中文计量单位,如“上午8点3刻”,不能写成“8h45min”。

计量单位符号一律用正体。
\subsection{外文字母的正、斜体用法}
按照GB3100~3102-86及GB7159-87的规定使用,即物理量符号、物理常量、变量符号用斜体,计量单位等符号均用正体。

\subsection{数字}
按国家语言文字工作委员会等七单位1987年发布的《关于出版物上数字用法的试行规定》,除习惯用中文数字表示的以外,一般均采用阿拉伯数字(参照附录4)。

\subsection{公式}
公式原则上应居中书写。若公式前有文字(如“解”、“假定”等),文字空两格写,公式仍居中写。公式末不加标点。

公式序号按章编排,如第1章第一个公式序号为“(1-1)”,附录2中的第一个公式为(2-1)等。

文中引用公式时,一般用“见式(1-1)”或“由公式(1-1)”。

公式中用斜线表示“除”的关系时应采用括号,以免含糊不清,如$a/(bcosx)$。
通常“乘”的关系在前,如$a cosx / b$而不写成$(a/b)cosx$。

\begin{equation}
    sigmoid(x)=\frac{1}{1+e^{-x}}
    \label{Eq:1}
\end{equation}

这里是正文中对公式的引用,由\gongshi{Eq:1}所示。

\subsection{插表}
表格不加左、右边线。

每个表格均应有表题(由表序和表名组成)。表序一般按章编排,如第1章第一个插表的序号为“表1-1”等。表序与表名之间空一格,表名中不允许使用标点符号,表名后不加标点。表题置于表上,用中、英文两种文字居中排写,中文在上,要求用5号字(参照附录5)。

表头设计应简单明了,尽量不用斜线。表头中可采用化学符号或物理量符号。

全表如用同一单位,则将单位符号移至表头右上角,加圆括号(参照附录5中例2)。

表中数据应准确无误,书写清楚。数字空缺的格内加横线“-”(占2个数字宽度)。表内文字或数字上、下或左、右相同时,采用通栏处理方式(参照附录5中例2),不允许用“〃”、“同上”之类的写法。

表内文字说明,起行空一格、转行顶格、句末不加标点。

\begin{table}[h]
    \bicaption{表格标题}{Name of table}
    \label{Tab:1}
    \centering
    \begin{tabular}{cccccc}
        \toprule
        A & B &A & B &A & B\\
        \midrule
        C & D & C & D & C & D\\
        C & D & C & D & C & D\\
        C & D & C & D & C & D\\
        C & D & C & D & C & D\\
        \bottomrule
    \end{tabular}
\end{table}


这里是正文中对表格的引用,如\biaoge{Tab:1}所示。

\subsection{插图}

插图应与文字紧密配合,文图相符,内容正确。选图要力求精练。

机械工程图:采用第一角投影法,严格按照GB4457~GB131-83《机械制图》标准规定。

电气图:图形符号、文字符号等应符合附录6所列有关标准的规定。

流程图:原则上应采用结构化程序并正确运用流程框图。

对无规定符号的图形应采用该行业的常用画法。

\begin{figure}[h]
    \centering
    \subcaptionbox{这是子图1 \label{Fig:1a}}[.4\textwidth]
    {\includegraphics[width=40mm]{img/hlju_logo.jpg}}
    \subcaptionbox{这是子图2 \label{Fig:1b}}[.4\textwidth]
    {\includegraphics[width=40mm]{img/hlju_logo.jpg}}
    \subcaptionbox{这是子图3 \label{Fig:1c}}[.4\textwidth]
    {\includegraphics[width=40mm]{img/hlju_logo.jpg}}
    \subcaptionbox{这是子图4 \label{Fig:1d}}[.4\textwidth]
    {\includegraphics[width=40mm]{img/hlju_logo.jpg}}
    \bicaption{中文插图标题}{Name of figures}
    \label{Fig:1}
\end{figure}

这里是正文中对图表的引用,如\tupian{Fig:1}所示。

\subsubsection{图题及图中说明}
每个图均应有图题(由图号和图名组成)。图号按章编排,如第1章第一个插图的图号为“图1-1”等。
图题置于图下,用中、英文两种文字居中书写,中文在上,要求用5号字。有图注或其它说明时应置于图题之上。
图名在图号之后空一格排写。引用图应注明出处,在图题右上角加引用文献号。
图中若有分图时,分图题置于分图之下,分图号用a)、b)等表示。

图中各部分说明应采用中文(引用的外文图除外)或数字项号,
各项文字说明置于图题之上(有分图题者,置于分图题之上)。

\subsubsection{插图编排}
插图之前,文中必须有关于本插图的提示,如“见图1-1”、“如图1-1所示”等。
插图与其图题为一个整体,不得拆开排写于两页。插图处的该页空白不够排写该图整体时,
则可将其后文字部分提前排写,将图移到次页最前面。

\subsubsection{坐标单位}
有数字标注的坐标图,必须注明坐标单位。

\subsubsection{论文原件中照片图及插图}
学位论文原件中的照片图均应是原版照片粘贴,不得采用复印方式。照片可为黑白或彩色,应主题突出、层次分明、清晰整洁、反差适中。照片采用光面相纸,不宜用布纹相纸。对金相显微组织照片必须注明放大倍数。

学位论文原件中的插图不得采用复印件。对于复杂的引用图,可采用数字化仪表输入计算机打印出来的图稿。
 
\subsection{参考文献}

参考文献编写项目和顺序规定如下:

a.专著、论文集、学位论文、报告

[序号] 主要责任者.文献题名[文献类型标识].出版地:出版者,出版年.起止页码.

b.期刊文章

[序号] 主要责任者.文献题名[文献类型标识].刊名,年,卷(期):起止页码.

c.论文集中的析出文献

[序号]析出文献主要责任者.析出文献题名[文献类型标识].原文献主要责任者(任选).原文献题名[文献类型标识].出版地:出版者,出版年.析出文献起止页码.

d.报纸文章

[序号] 主要责任者.文献题名[文献类型标识].报纸名,出版日期(版次).

e.国际、国家标准

[序号] 标准编号,标准名称[文献类型标识].

f.专利

[序号] 专利所有者.专利题名[文献类型标识].专利国别:专利号,出版日期.

g.电子文献

[序号] 主要责任者.电子文献题名[电子文献及载体类型标识].电子文献的出处或可获得地址,
发表或更新日期/引用日期(任选).

h.各种未定义类型的文献

[序号] 主要责任者.文献题名[Z].出版地:出版者,出版年.

\begin{table}[h]
    \centering
    \caption{参考文献特殊标识}
    \begin{tabular}{|c|c|c|c|c|c|c|c|c|}
        \hline
        参考文献类型&专著&论文集&报纸文章&期刊文章&学位论文&报告&标准&专利\\
        \hline
        文献类型标识&[M]&[C]&[N]&[J]&[D]&[R]&[S]&[P]\\
        \hline
    \end{tabular}
\end{table}

对于专著、论文集中的析出文献,其文献类型标识建议采用单字母“A”;对于其他未说明的文献类型,建议采用单字母“Z”。

对于数据库(database)、计算机程序(computer program)及电子公告(electronic bulletin board)等电子文献类型的参考文献,建议以下列双字母作为标识:
\begin{table}[h]
    \centering
    \begin{tabular}{|c|c|c|c|}
        \hline
        电子参考文献类型&数据库&计算机程序&电子公告\\
        \hline
        电子文献类型标识&[DB]&[CP]&[EB]\\
        \hline
    \end{tabular}
\end{table}

\subsection{附录}

对需要收录于学位论文中且又不适合书写于正文中的附加数据、资料、详细公式推导等有特色的内容,可作为附录排写,序号采用“附录1”、“附录2”等。

\subsection{攻读学位期间发表论文}
要求博士学位论文必须有该项内容,书写格式与参考文献相同。硕士学位论文必要时有该项内容。

\subsection{索引}

为便于检索文中内容,可编制索引置于论文之后(根据需要决定是否设置)。
索引以论文中的专业词语为检索线索,指出其相关内容的所在页码。
索引用中、英两种文字书写,中文在前。中文按各词汉语拼音第一个字母排序,英文按该词第一个英文字母排序。
索引示例参照附录8。



