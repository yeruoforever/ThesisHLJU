
\section*{攻读学位期间发表的学术论文}
\addcontentsline{toc}{section}{攻读学位期间发表的学术论文}

\renewcommand\refname{}

学位论文后应列出研究生在攻读学位期间发表的与学位论文内容相关的学术论文
(含已录用,并有录用通知书的学术论文。录用通知书中应明确说明论文的发表卷、期号。)。
攻读学位期间所获得的科研成果可单做一项列出。
与学位论文无关的学术论文不宜在此列出。
我校对博士生攻读学位期间发表学术论文的要求见《黑龙江大学关于研究生学位论文答辩资格审查暂行规定》。 

这是我的第一篇论文\cite{xuan_dynamic_2022}

这是我的第二篇论文\cite{10.1093/bib/bbac009}

\begin{thebibliography}{2}
    \providecommand{\natexlab}[1]{#1}
    \providecommand{\url}[1]{#1}
    \expandafter\ifx\csname urlstyle\endcsname\relax\else
      \urlstyle{same}\fi
    \expandafter\ifx\csname href\endcsname\relax
      \DeclareUrlCommand\doi{\urlstyle{rm}}
      \def\eprint#1#2{#2}
    \else
      \def\doi#1{\href{https://doi.org/#1}{\nolinkurl{#1}}}
      \let\eprint\href
    \fi
    
    \bibitem[Xuan et~al.(2022)Xuan, Cui, Zhang, Zhang, Wang, Nakaguchi, and
      Duh]{xuan_dynamic_2022}
    XUAN P, CUI H, ZHANG H, et~al.
    \newblock Dynamic graph convolutional autoencoder with node-attribute-wise
      attention for kidney and tumor segmentation from {CT}
      volumes\allowbreak[J/OL].
    \newblock Knowledge-Based Systems, 2022, 236: 107360.
    \newblock
      \url{https://www.sciencedirect.com/science/article/pii/S0950705121006225}.
    \newblock DOI: \doi{https://doi.org/10.1016/j.knosys.2021.107360}.
    
    \bibitem[Zhang et~al.(2022)Zhang, Cui, Zhang, Cao, and
      Xuan]{10.1093/bib/bbac009}
    ZHANG H, CUI H, ZHANG T, et~al.
    \newblock {Learning multi-scale heterogenous network topologies and various
      pairwise attributes for drug–disease association
      prediction}\allowbreak[J/OL].
    \newblock Briefings in Bioinformatics, 2022.
    \newblock \url{https://doi.org/10.1093/bib/bbac009}.
    
    \end{thebibliography}