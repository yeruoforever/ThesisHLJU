\section{绪论}
\subsection{研究目的及意义}

\pagebreak

\section{课题研究理论基础}
\subsection{生物信息学相关知识}
\subsubsection{LncRNA特点}

这是我的第一篇论文\cite{xuan_dynamic_2022}

这是我的第二篇论文\cite{10.1093/bib/bbac009}


\begin{figure}[h]
    \centering
    \subcaptionbox{这是子图1 \label{Fig:1a}}[.4\textwidth]
        {\includegraphics[width=40mm]{img/hlju_logo.jpg}}
    \subcaptionbox{这是子图2 \label{Fig:1b}}[.4\textwidth]
        {\includegraphics[width=40mm]{img/hlju_logo.jpg}}
    \subcaptionbox{这是子图3 \label{Fig:1c}}[.4\textwidth]
        {\includegraphics[width=40mm]{img/hlju_logo.jpg}}
    \subcaptionbox{这是子图4 \label{Fig:1d}}[.4\textwidth]
        {\includegraphics[width=40mm]{img/hlju_logo.jpg}}
    \bicaption{中文图标标题}{Name of figures}
    \label{Fig:1}
\end{figure}

\begin{table}[h]
    \bicaption{表格标题}{Name of table}
    \label{Tab:1}
    \centering
    \begin{tabular}{cccccc}
        \toprule
        A & B &A & B &A & B\\
        \midrule
        C & D & C & D & C & D\\
        C & D & C & D & C & D\\
        C & D & C & D & C & D\\
        C & D & C & D & C & D\\
        \bottomrule
    \end{tabular}
\end{table}


\begin{equation}
    sigmoid(x)=\frac{1}{1+e^{-x}}
    \label{Eq:1}
\end{equation}

这里是正文中对图表的引用,如\tupian{Fig:1}所示。

这里是正文中对表格的引用,如\biaoge{Tab:1}所示。

这里是正文中对公式的引用,如\gongshi{Eq:1}所示。
