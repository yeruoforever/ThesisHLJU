\documentclass[12pt]{ctexart}

% \usepackage[UTF8]{ctex}
\usepackage{graphicx}
\usepackage{setspace}
\usepackage{fancyhdr}
\usepackage{ulem}
\usepackage{titlesec}
\usepackage{titletoc}
\usepackage[backend=biber,style=gb7714-2015,gbnamefmt=givenahead,gbmedium=false,gbpunctin=false,gbalign=left]{biblatex}
\usepackage{caption}
\usepackage{bicaption}
\usepackage{subcaption}
\usepackage{amsmath}
\usepackage{amsfonts}
\usepackage{booktabs}
\usepackage{multirow}
\usepackage{makecell}
\usepackage{bm}
\usepackage[a4paper,total={150mm,215mm},top=40mm,inner=3cm,outer=3cm,bottom=35mm,twoside,headsep=5mm,footskip=5.25mm]{geometry}
\usepackage{longtable}

\newcommand{\lwtm}{论文题目}
% 申请人
\newcommand{\shenqingren}{申请人姓名}
% 学号
\newcommand{\xuehao}{学号}
% 培养单位
\newcommand{\pydw}{培养名}
% 学科专业名
\newcommand{\xkzy}{学科专业名}
% 研究方向名称
\newcommand{\yjfx}{研究方向名称}
% 指导教师
\newcommand{\zdjs}{指导教师}
% 完成日期
\newcommand{\wcrq}{\today}

% 黑龙江大学研究生毕业论文要求
% A4纸
% 上边距 4.0cm
% 下边距 3.5cm
% 内侧 3cm
% 外侧 3cm
% 版心大小 150mm x 215mm (加页眉页码 150mm x 235mm)

% 页眉 3cm
% 页脚 2.75cm

\newdimen\doublelineskip % 两横线间的距离
\setlength\doublelineskip{.3mm}
\newdimen\headrulewidthA % 第一条线的宽度
\setlength\headrulewidthA{.2mm}
\newdimen\headrulewidthB % 第二条线的宽度
\setlength\headrulewidthB{0.8mm}


\fancypagestyle{headings}{
  \renewcommand\headrulewidth{0.4pt}
}
\fancypagestyle{doubleline}[headings]{
  \renewcommand\headrule{%
    \hrule height\headrulewidthA width\headwidth%
    \vskip \doublelineskip%
    \hrule height\headrulewidthB width\headwidth}
    }
    
%\pagestyle{headings}
\pagestyle{doubleline}



% \pagestyle{fancy}
%页眉上字号用5号黑体。
\fancyhead{}
\fancyhead[CO]{\zihao{5}\heiti\leftmark}
\fancyhead[CE]{\zihao{5}\heiti 黑龙江大学硕士学位论文}

% 页下注字号用小5号宋体.
\fancyfoot[CO,CE]{\zihao{-5}\songti\thepage}

% 学位论文的字号正文用小4号宋体,英文为Times New Roman
\setmainfont{Times New Roman}
% 1.5倍行距。
% \onehalfspacing
\setstretch{1.614}
% \punctstyle{banjiao}
% 各章题序及标题用小2号黑体,
% 各节的一级题序及标题用小3号黑体,
% 各节的二级题序及标题用4号黑体,
% 各节的三级题序及标题用小4号黑体,款项均采用小4号黑体,其余用小4号宋体。
\ctexset{
  section={
    name={第,章},
    format += \zihao{-2}\heiti\mdseries
  },
  subsection={
    format += \zihao{-3}\heiti\mdseries,
    indent = 0\ccwd
  },
  subsubsection={
    format += \zihao{4}\heiti\mdseries,
    indent = 0\ccwd
  }
}

\newcommand{\kuanxiang}[1]{{\heiti \zihao{-4} #1}\ }

% \titlecontents{标题名}
% [左间距]
% {标题格式}
% {标题标志}
% {无序号标题}
% {指引线与页码}
% [下间距]

% 正文章节题目(要求编到第3级标题,即×.×.×。一级标题顶格书写,二级标题缩进一格,三级标题缩进两格。)
\titlecontents{section}
              [3.8em]
              {\songti \zihao{-4} \bfseries}
              {\contentslabel{3.5em}}
              {\hspace*{-3.5em}}
              {\titlerule*[0.3pc]{.}\contentspage}
\titlecontents{subsection}
              [4em]
              {\songti \zihao{-4} }
              {\contentslabel{1.8em}}
              {\hspace*{-1.8em}}
              {\titlerule*[0.3pc]{.}\contentspage}
\titlecontents{subsubsection}
              [6.5em]
              {\songti \zihao{-4} }
              {\contentslabel{2.5em}}
              {\hspace*{-2.5em}}
              {\titlerule*[0.3pc]{.}\contentspage}





% 每个图均应有图题(由图号和图名组成)。图号按章编排,
% 如第1章第一个插图的图号为“图1-1”等。图题置于图下,用中、英文两种文字居中书写,
% 中文在上,要求用5号字。有图注或其它说明时应置于图题之上。
% 图名在图号之后空一格排写。引用图应注明出处,在图题右上角加引用文献号。
% 图中若有分图时,分图题置于分图之下,分图号用a)、b)等表示。

\captionsetup[figure]{labelsep=space}
\captionsetup[figure][bi-second]{name=Figure}
\captionsetup[table]{labelsep=space}
\captionsetup[table][bi-second]{name=Table}
\captionsetup[subfigure]{labelformat=simple}

\renewcommand{\thefigure}{\thesection-\arabic{figure}}
\renewcommand{\thetable}{\thesection-\arabic{table}}
\renewcommand{\thesubfigure}{\alph{subfigure})}


\renewcommand{\captionfont}{\zihao{5}}

% 公式
% 公式原则上应居中书写。若公式前有文字(如“解”、“假定”等),文字空两格写,公式仍居中写。公式末不加标点。
% 公式序号按章编排,如第1章第一个公式序号为“(1-1)”,附录2中的第一个公式为(2-1)等。
% 文中引用公式时,一般用“见式(1-1)”或“由公式(1-1)”。

\numberwithin{equation}{section}
\renewcommand{\theequation}{\arabic{section}-\arabic{equation}}
\numberwithin{figure}{section}
\renewcommand{\thefigure}{\arabic{section}-\arabic{figure}}
\newcommand{\gongshi}[1]{公式\eqref{#1}}
\newcommand{\tupian}[1]{图\ref{#1}}
\newcommand{\biaoge}[1]{表\ref{#1}}


\addbibresource[location=local]{content/references.bib}


\begin{document}

% 正文小4号宋体
\songti
\zihao{-4}

\begin{titlepage}
  \onehalfspacing
  \hbox{ \hsize 100mm
    \vbox{\includegraphics[width=37.2mm,height=34.8mm]{img/hlju_logo.jpg}}
    \raise 8mm
    \vbox{
      \heiti
      \zihao{-4}
      \bfseries 
      \makebox[3em][s]{分类号} \hspace{1ex} \dotuline{\makebox[4em][s]{}}

      \makebox[3em][s]{U D C} \hspace{1ex} \dotuline{\makebox[4em][s]{}}

      \makebox[3em][s]{密级} \hspace{1ex} \dotuline{\makebox[4em][c]{\makebox[3em][s]{公开}}}

    }
  }

\vspace{8mm}

\begin{center}
  \includegraphics[width=27.3\ccwd,height=6.\ccwd]{img/school_name.png}

  \vspace{8mm}

  {\songti \zihao{2} 硕士研究生学位论文}

  \vspace{8mm}

  {\heiti \zihao{2} \bfseries \lwtm}

  \vspace{30mm}

  {
    \songti \zihao{-3} \bfseries
    \begin{tabular}{ll}
      \makebox[4em][s]{申请人:}& \shenqingren\\
      \makebox[4em][s]{学号:}&  \xuehao\\
      \makebox[4em][s]{培养单位:}& \pydw\\
      \makebox[4em][s]{学科专业:}& \xkzy\\
      \makebox[4em][s]{研究方向:}& \yjfx\\
      \makebox[4em][s]{指导教师:}& \zdjs\\
      \makebox[4em][s]{完成日期:}& \wcrq\\
    \end{tabular}
    
  }
\end{center}
  
\end{titlepage}

\pagenumbering{Roman}
\fancyhead[CO]{\zihao{5}\heiti 中文摘要}
\section*{中文摘要}
\addcontentsline{toc}{section}{中文摘要}
这是中文摘要。

~\\
\noindent {\heiti \zihao{-4} 关键词:\ }(词) ;(词) ;… ;(词)
\pagebreak
\fancyhead[CO]{\zihao{5}\heiti Abstract}
\section*{Abstract}
\addcontentsline{toc}{section}{Abstract}
This is english.
\pagebreak
\fancyhead[CO]{\zihao{5}\heiti 目录}
\tableofcontents
\section*{符号说明}

我的符号说明

\pagebreak

\pagenumbering{arabic}
% main body
\fancyhead[CO]{\zihao{5}\heiti\leftmark}
\section{绪论}
\subsection{研究目的及意义}


这是我的第一篇论文\cite{xuan_dynamic_2022}

这是我的第二篇论文\cite{10.1093/bib/bbac009}

\pagebreak

\section{课题研究理论基础}
\subsection{生物信息学相关知识}
\subsubsection{LncRNA特点}

\section*{附录}
\addcontentsline{toc}{section}{附录}
对需要收录于学位论文中且又不适合书写于正文中的附加数据、资料、详细公式推导等有特色的内容,可作为附录排写,序号采用“附录1”、“附录2”等。

\renewcommand{\leftmark}{结论}
\section*{结论}
\addcontentsline{toc}{section}{结论}
可以总结性地说明学位论文的最终研究成果及其价值,
同时,可以直接表明论文中尚待进一步研究的不足之处。

结论应当体现作者更深的认识,且是从全篇论文的全部材料出发,经过推理、判断、归纳等逻辑分析过程而得到的新的学术总观念、总见解。
 \pagebreak
\fancyhead[CO]{\zihao{5}\heiti 参考文献}
\begin{spacing}{1}
  \printbibliography[heading=bibliography,title=参考文献]
\end{spacing}
\addcontentsline{toc}{section}{参考文献}

\section*{附录}
\addcontentsline{toc}{section}{附录}
对需要收录于学位论文中且又不适合书写于正文中的附加数据、资料、详细公式推导等有特色的内容,可作为附录排写,序号采用“附录1”、“附录2”等。

\fancyhead[CO]{\zihao{5}\heiti 致谢}
\section*{致谢}
\addcontentsline{toc}{section}{致谢}
对导师和给予指导或协助完成学位论文工作的组织和个人表示感谢。
内容应简洁明了、实事求是。对课题给予资助者应予感谢。
\fancyhead[CO]{\zihao{5}\heiti 攻读学位期间发表的学术论文}

\section*{攻读学位期间发表的学术论文}
\addcontentsline{toc}{section}{攻读学位期间发表的学术论文}
我的成果

\section*{独创性声明}
\fancyhead[CO]{\zihao{5}\heiti 独创性声明}
\addcontentsline{toc}{section}{独创性声明}

本人声明所呈交的学位论文是本人在导师指导下进行的研究工作及取得的研究成果.
据我所知, 除了文中特别加以标注和致谢的地方外,
论文中不包含其他人已经发表或撰写过的研究成果,
也不包含为获得\underline{\makebox[6\ccwd][c]{黑龙江大学}}或其他教育机构的
学位或证书而使用过的材料。

\vspace{1cm} \noindent 学位论文作者签名: \hspace{11\ccwd}
签字日期:\hspace{2\ccwd}年\hspace{1\ccwd}月\hspace{1\ccwd}日
\vspace{1cm}

\section*{学位论文版权使用授权书}
本人完全了解\underline{\makebox[6\ccwd][c]{黑龙江大学}}有关保留、使用学位论文的规定,
同意学校保留并向国家有关部门或机构送交论文的复印件和电子版,允许论文被查阅和借阅。
本人授权\underline{\makebox[6\ccwd][c]{黑龙江大学}}
可以将学位论文的全部或部分内容编入有关数据库进行检索,
可以采用影印、缩印或其他复制手段保存、汇编本学位论文。

\vspace{1cm}
\noindent 学位论文作者签名: \hspace{11\ccwd} 导师签名:

\noindent 签字日期:\hspace{2\ccwd}年\hspace{1\ccwd}月\hspace{1\ccwd}日
\hspace{7.7\ccwd}
签字日期:\hspace{2\ccwd}年\hspace{1\ccwd}月\hspace{1\ccwd}日

\vspace{1cm}
\noindent 学位论文作者毕业后去向:

\noindent 工作单位:\hspace{15.3\ccwd} 电话:

\noindent 通讯地址:\hspace{15.3\ccwd} 邮编:


\end{document}